\chapter{Исследовательская часть}
В данном разделе будут приведены примеры работы программа, а также
проведен сравнительный анализ алгоритмов для разных ключей на основе полученных данных.

\section{Технические характеристики}



Технические характеристики устройства, на котором выполнялись эксперименты:



\begin{itemize}

	
	\item операционная система: Ubuntu 20.04.3 \cite{ubuntu} Linux \cite{linux} x86\_64;

	
	\item память: 8 ГБ;

	
	\item процессор: 11th Gen Intel® Core™ i5-1135G7 @ 2.40GHz \cite{intel}.

	
\end{itemize}


Эксперименты проводились на ноутбуке, включенном в сеть электропитания. Во время экспериментов ноутбук был нагружен только встроенными приложениями окружения, а также непосредственно системой тестирования.


\section{Демонстрация работы программы}

На рисунках \ref{img:p1} и \ref{img:p2} представлены результаты работы программы.

\img{100mm}{p1}{Пример 1 работы программы}
\img{100mm}{p2}{Пример 2 работы программы}
\clearpage

\section{Время выполнения алгоритмов}
Как было сказано выше, для замера времени выполнения части кода используется функция process\_time() из библиотеки time \cite{pythonlangtime}. 

На рисунке \ref{img:grah} представлен график зависимости времени поиска от
индекса ключа словаря, для построения которого использовались данные
о времени поиска каждого элемента в словаре(2000 элемент). Индекс ключа
указан на горизонтальной оси.

\img{100mm}{grah}{Время работы алгоритмов для поиска каждого ключа словаря}

На рисунке \ref{img:grah2} представлен график на меньшем количестве точек для большей наглядности.

\img{100mm}{grah2}{Время работы алгоритмов для поиска каждого ключа словаря}
\newpage

\section{Количество сравнений}
В ходе эксперимента было подсчитано количество сравнений для каждого ключа в словаре, и на основе полученных данных составлены гистограммы.

Гистограммы составлены двух видов для каждого алгоритма:
\begin{itemize}
	\item ключи отсортированы в лексикографическом порядке (для словаря, рассматриваемого в данной работе);
	\item ключи расположены в том порядке, в котором они идут в самом словаре.	
\end{itemize}

Гистограммы для алгоритма поиска в словаре полным перебором представлены на рисунке \ref{img:bfs}.

\img{250mm}{bfs}{Поиск с помощью полного перебора}


Гистограммы для алгоритма бинарного поиска в словаре представлены на рисунке \ref{img:bs}.

\img{250mm}{bs}{Бинарный поиск}

Гистограммы для алгоритма поиска в словаре с помощью разбиения на сегменты представлены на рисунке \ref{img:ss}.

\img{250mm}{ss}{С помощью сегментов}
\clearpage

\section{Вывод}
В результате эксперимента и полученных графиков и гистограмм, приведенных выше, видно, что самый медленный алгоритм - алгоритм полного перебора. Время в нём растет линейно и увеличивается с увеличением индекса элемента словаря. Также растет количество сравнений, что напрямую зависит от времени работы программы.

Алгоритм бинарного поиска и алгоритм поиска с помощью сегментов тратят примерно одинаковое количество времени, при этом они требуют дополнительных расходов времени на $  $подготовку данных к работе с алгоритмом, но эти расходы малы.