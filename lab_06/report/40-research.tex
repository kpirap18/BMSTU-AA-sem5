\chapter{Исследовательская часть}
В данном разделе будет произведено сравнение вышеизложенного алгоритма (однопоточная и многопоточная реализация).

\section{Технические характеристики}


Технические характеристики устройства, на котором выполнялись эксперименты:


\begin{itemize}

	\item операционная система: Ubuntu 20.04.3 \cite{ubuntu} Linux \cite{linux} x86\_64;

	\item память: 8 ГБ;

	\item процессор: 11th Gen Intel® Core™ i5-1135G7 @ 2.40GHz \cite{intel}.

\end{itemize}

Эксперименты проводились на ноутбуке, включенном в сеть электропитания. Во время экспериментов ноутбук был нагружен только встроенными приложениями окружения, а также непосредственно системой тестирования.

\section{Пример выполнения программы}

На рисунках \ref{img:prim} \ref{img:prim2} представлен результат работы программы.

В начале выводится матрица смежности, которая поступила на вход.

Далее печатаются результаты работы алгоритма полного перебора и муравьиного алгоритма. 

На изображениях графа представлены результаты работы в виде путей. Зеленым отмечен результат работы алгоритма полного перебора, красным -- муравьиный алгоритм.

Параметры для муравьиного алгоритма взяты из раздела 4.5 данной главы.

\img{80mm}{prim}{Пример работы программы}

\img{40mm}{prim2}{Пример работы программы}

\newpage

\section{Время выполнения алгоритмов}

Для проведения временного анализа, программа запускалась на различных размерах графов.

Результаты сравнения времени работы алгоритмов полного перебора и муравьиного алгоритма, приведены в таблице \ref{tbl:only4}.

Параметры для муравьиного алгоритма взяты из раздела 4.5 данной главы.

\begin{table}[h]
	\caption{Время (такт) выполнения полного перебора и муравьиного алгоритма}
	\label{tbl:only4}
	\begin{center}
		\begin{tabular}{|c|c|c|c|c|}
			\hline
			& \multicolumn{2}{c|}{\bfseries Тип алгоритма}           \\ \cline{2-3}
			\bfseries Размер & \bfseries Полный перебор & \bfseries Муравьиный 
			\csvreader{inc/csv/test.csv}{}
			{\\\hline \csvcoli&\csvcolii&\csvcoliii}
			\\\hline
		\end{tabular}
	\end{center}
\end{table}
\newpage 

На рисунках \ref{img:f2} представлен график зависимости времени от размера матрицы смежности для алгоритма полного перебора и муравьиного алгоритма.

Параметры для муравьиного алгоритма взяты из раздела 4.5 данной главы.

\img{90mm}{f2}{Замеры времени работы}
\newpage

\section{Постановка эксперимента}
В муравьином алгоритме вычисления производятся на основе настраиваемых параметров.
Рассмотрим три класса данных и подберем к ним параметры, при которых метод даст точный результат при минимальном количестве итераций (итерациями считаются кол-во дней).

Будем рассматривать матрицы размерности $10\times10$, так как иначе получить точный результат будет довольно просто.

В качестве первого класса данных будет матрица смежности, в которой все значения незначительно отличаются друг от друга, например, в диапазоне $[1, 10]$.

Вторым классом будет матрица, где значения могут отличаться на большей значение, например $[3000, 8000]$.

Третьим классом будет матрица, где значения почти все одинаковые, кроме пару путей, то есть самый короткий путь определен однозначно.

Будем запускать муравьиный алгоритм для всех значений $\alpha, \rho\in[0, 1]$, с шагом $= 0.1$.

В результате тестирования будет выведена таблица со значениями $\alpha, \rho$, $days$, $\Delta L_{1}$, $\Delta L_{2}$, $\Delta L_{3}$, где
\begin{itemize}
	\item $days$ — количество дней, данных для решения задачи;
	\item $\Delta L_{1}$ - разность эталонного решения задачи и решения муравьиного алгоритма с данными параметрами на первом классе данных.
	\item $\Delta L_{2}$ - разность эталонного решения задачи и решения муравьиного алгоритма с данными параметрами на втором классе данных.
	\item $\Delta L_{3}$ - разность эталонного решения задачи и решения муравьиного алгоритма с данными параметрами на третьем классе данных.
	\item $\alpha, \beta, \rho$ — настроечные параметры.
\end{itemize}

\subsubsection{Класс данных 1}
Матрица смежности для данного класса данных \eqref{matrix}:

\begin{equation}
\label{matrix}
M = \begin{pmatrix}
0 & 1 & 6 & 7 & 5 & 4 & 4 & 2 & 1 & 2\\
1 & 0 & 1 & 2 & 1 & 3 & 2 & 1 & 2 & 1\\
6 & 1 & 0 & 1 & 2 & 8 & 2 & 1 & 6 & 5\\
7 & 2 & 1 & 0 & 1 & 3 & 1 & 4 & 1 & 8\\
5 & 1 & 2 & 1 & 0 & 2 & 2 & 6 & 1 & 7\\
4 & 3 & 8 & 3 & 2 & 0 & 3 & 8 & 1 & 9\\
4 & 2 & 2 & 1 & 2 & 3 & 0 & 1 & 8 & 2\\
2 & 1 & 1 & 4 & 6 & 8 & 1 & 0 & 7 & 3\\
1 & 2 & 6 & 1 & 1 & 1 & 8 & 7 & 0 & 1\\
2 & 1 & 5 & 8 & 7 & 9 & 2 & 3 & 1 & 0
\end{pmatrix}
\end{equation}


\subsubsection{Класс данных 2}
Матрица смежности для данного класса данных \eqref{matrix1}:
\begin{equation}
\label{matrix1}
M = \begin{pmatrix}
0  &3000  &4599  &7000  &3412  &3388  &4454  &3498  &3541  &3498 \\
3000  &0  &3607  &3000  &3432  &3033  &2339  &5621  &3984  &3121 \\
4599  &3607  &0  &3490  &3988  &3834  &3121  &8731  &4454  &3245 \\
7000  &3000  &3490  &0  &3890  &3444  &3245  &3341  &2339  &5621 \\
3412  &3432  &3988  &3890  &0  &3004  &3322  &7863  &3121  &8731 \\ 
3388  &3033  &3834  &3444  &3004  &0  &3238  &3468  &3245  &3341 \\
4454  &2339  &3121  &3245  &3322  &3238  &0  &3218  &3322  &7863 \\
3498  &5621  &8731  &3341  &7863  &3468  &3218  &0  &3238  &3468 \\
3541  &3984  &4454  &2339  &3121  &3245  &3322  &3238  &0  &3218\\
3498  &3121  &3245  &5621  &8731  &3341  &7863  &3468  &3218  &0 
\end{pmatrix}
\end{equation}


\subsubsection{Класс данных 3}
Матрица смежности для данного класса данных \eqref{matrix3}:
\begin{equation}
\label{matrix3}
M = \begin{pmatrix}
0    &111  &111  &111  &111  &111  &111  &112  &111  &111 \\
111  &0    &112  &111  &111  &111  &111  &111  &111  &111 \\
111  &112  &0    &111  &111  &112  &111  &111  &111  &111 \\
111  &111  &111  &0    &111  &111  &111  &111  &111  &111 \\
111  &111  &111  &111  &0    &111  &111  &111  &111  &111 \\
111  &111  &112  &111  &111  &0    &111  &111  &111  &111 \\
111  &111  &111  &111  &111  &111  &0    &111  &111  &111 \\
112  &111  &111  &111  &111  &111  &111  &0    &112  &111 \\
111  &111  &111  &111  &111  &111  &111  &112  &0    &111 \\
111  &111  &111  &111  &111  &111  &111  &111  &111  &0 \\
\end{pmatrix}
\end{equation}

Полные результаты данного эксперимента приведены в приложении.

\section{Вывод}
В результате работы в данном разделе можно сделать вывод о том, что сравнительный анализ времени работы двух реализованных алгоритмов указал на факт того, что на всех рассмотренных значениях размерности матрицы смежности муравьиный алгоритм работает лучше, чем алгоритм полного перебора. 

Из таблицы \ref{tbl:only4} можно увидеть, что на минимальной рассмотренной размерности матрицы (от $3\time3$ до $7\time7$) муравьиный алгоритм работает быстрее алгоритма полного перебора на $\approx 5 - 8\%$. На конечных значениях (от $8\time38$ до $10\time10$) разрыв увеличивается, и муравьиный алгоритм работает на $\approx 50 - 60\%$ быстрее.

Из графика, предоставленного на рисунке \ref{img:f2}, видно, что муравьиный алгоритм до размерности матрицы смежности, равной 7, сравним по скорости работы алгоритма полного перебора. Видное преимущество муравьиного алгоритма будет заметно, начиная с размерности матрицы смежности, равной 8.

На основе проведенной параметризации для двух классов данных можно сделать следующие выводы.
\begin{enumerate}
	\item Для класса данных, предоставленного в уравнении \ref{matrix} в качестве матрицы смежности, наилучшими наборами стали $(\alpha = 0.6, 0.8, \rho = 0.7, 0.1, 0.4)$, так как они показали наиболее стабильные результаты, то $\Delta L_{1} = 0$.
	\item Для класса данных, предоставленного в уравнении \ref{matrix1} в качестве матрицы смежности, наилучшими наборами стали $(\alpha = 0.6, 0.7, \rho = 0.7, 0.3)$. При этих параметрах $\Delta L_{2} = 0$.
	\item Для класса данных, предоставленного в уравнении \ref{matrix3} в качестве матрицы смежности, наилучшими наборами стали $(\alpha = 0.5, 0.6, \rho = 0.5, 0.7)$. При этих параметрах $\Delta L_{3} = 0$.
\end{enumerate}

Из результатов выше можно сделать вывод о том, что наилучшим набором для всех трех классов данных является набор $(\alpha = 0.6, \rho = 0.7)$, при этом $tmax$ может равняться любому значению. 

Также следует отметить, что из представленных таблицы и графике можно сделать вывод, что трудоемкость алгоритма полного перебора равна $O(n!)$, а муравьиного алгоритма равна $O(tmax * n^3)$ \cite{Ulianov}.