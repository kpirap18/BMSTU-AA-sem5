\chapter{Технологическая часть}

В данном разделе будут приведены требования к программному обеспечению, средства реализации и листинги кода.

\section{Требования к ПО}

К программе предъявляется ряд требований:
\begin{itemize}
	\item на вход подаётся массив сравнимых элементов (целые числы);
	\item на выходе — тот же массив, но в отсортированном порядке.
\end{itemize}

\section{Средства реализации}

В качестве языка программирования для реализации данной лабораторной работы был выбран ЯП Python \cite{pythonlang}. 

Данный язык достаточно удобен и гибок в использовании. 

Время работы алгоритмов было замерено с помощью функции process\_time() из библиотеки time \cite{pythonlangtime}

\section{Сведения о модулях программы}
Программа состоит из двух модулей:
\begin{enumerate}
	\item main.py - главный файл программы, в котором располагаются коды всех алгоритмов и меню;
	\item test.py - файл с замерами времени для графического изображения результата.
\end{enumerate}


\section{Листинг кода}


\begin{lstlisting}[label=lst:vyb_sort,caption=Алгоритм сортировки выбором]
def selection_sort(arr):
	for i in range(0, len(arr) - 1):
		min = i
		for j in range(i + 1, len(arr)):
			if arr[j] < arr[min]:
				min = j
		arr[i], arr[min] = arr[min], arr[i]
	return arr
\end{lstlisting}

\begin{lstlisting}[label=lst:vst_sort,caption= Алгоритм сортировки вставками]
def insertion_sort(arr):
	for i in range(1, len(arr)):
		j = i - 1
		key = arr[i]
		
		while j >= 0 and arr[j] > key:
			arr[j + 1] = arr[j]
			j -= 1
		arr[j + 1] = key
	return arr	
\end{lstlisting}

\begin{lstlisting}[label=lst:sheyker_sort,caption=Алгоритм сортировки перемешиванием (или сортировка шейкер)]
def shaker_sort(arr):
	length = len(arr)
	swapped = True
	start_index = 0
	end_index = length - 1
	
	while (swapped == True):
		swapped = False
		
		for i in range(start_index, end_index):
			if (arr[i] > arr[i + 1]):
				arr[i], arr[i + 1] = arr[i + 1], arr[i]
				swapped = True
		
		if (not (swapped)):
			break
		
		swapped = False
		end_index = end_index - 1
		
		for i in range(end_index - 1, start_index - 1, -1):
			if (arr[i] > arr[i + 1]):
				arr[i], arr[i + 1] = arr[i + 1], arr[i]
				swapped = True
			
		start_index = start_index + 1
	return arr
\end{lstlisting}

\section{Функциональные тесты}

В таблице \ref{tbl:functional_test} приведены тесты для функций, реализующих алгоритмы сортировки. Тесты пройдены успешно.


\begin{table}[h]
	\begin{center}
		\caption{\label{tbl:functional_test} Функциональные тесты}
		\begin{tabular}{|c|c|c|}
			\hline
			Входной массив & Ожидаемый результат & Результат \\ 
			\hline
			$[1,2,3,4]$ & $[1,2,3,4]$  & $[1,2,3,4]$\\
			$[5,4,3,2,1]$  & $[1,2,3,4,5]$ & $[1,2,3,4,5]$\\
			$[3,2,-5,0,1]$  & $[-5,0,0,2,3]$  & $[-5,0,0,2,3]$\\
			$[4]$  & $[4]$  & $[4]$\\
			$[]$  & $[]$  & $[]$\\
			\hline
		\end{tabular}
	\end{center}
\end{table}


\section*{Вывод}

Были разработаны схемы всех трех алгоритмов сортировки. Для каждого из них были рассчитаны и оценены лучшие и худшие случаи.
