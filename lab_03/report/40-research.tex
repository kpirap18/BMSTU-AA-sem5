\chapter{Исследовательская часть}

В данном разделе будут приведены примеры работы программ, постановка эксперимента и сравнительный анализ алгоритмов на основе полученных данных.

\section{Технические характеристики}

Технические характеристики устройства, на котором выполнялось тестирование:

\begin{itemize}
	\item Операционная система: Ubuntu 20.04.3 \cite{ubuntu} Linux \cite{linux} x86\_64.
	\item Память: 8 GiB.
	\item Процессор: 11th Gen Intel® Core™ i5-1135G7 @ 2.40GHz \cite{intel}.
\end{itemize}

Тестирование проводилось на ноутбуке, включенном в сеть электропитания. Во время тестирования ноутбук был нагружен только встроенными приложениями окружения, а также непосредственно системой тестирования.

\section{Демонстрация работы программы}

На рисунке \ref{img:primer} представлен результат работы программы.

\img{80mm}{primer}{Пример работы программы}

\section{Время выполнения алгоритмов}

Алгоритмы тестировались при помощи функции process\_time() из библиотеки time языка Python. Данная функция всегда возвращает значения времени, а имеено сумму системного и пользовательского процессорного времени текущего процессора, типа float в секундах.

Контрольная точка возвращаемого значения не определна, поэтому допустима только разница между результатами последовательных вызовов.


\begin{table}[h]
	\begin{center}
		\caption{\label{tab:time}Результаты замеров времени.}
		\begin{tabular}{|c|c|c|c|c|}
			
			
		\end{tabular}
	\end{center}
\end{table}

\section*{Вывод}



