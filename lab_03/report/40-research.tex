\chapter{Исследовательская часть}

В данном разделе будут приведены примеры работы программ, постановка эксперимента и сравнительный анализ алгоритмов на основе полученных данных.

\section{Технические характеристики}

Технические характеристики устройства, на котором выполнялось тестирование, следующие.

\begin{itemize}
	\item Операционная система: Ubuntu 20.04.3 \cite{ubuntu} Linux \cite{linux} x86\_64.
	\item Память: 8 GiB.
	\item Процессор: 11th Gen Intel® Core™ i5-1135G7 @ 2.40GHz \cite{intel}.
\end{itemize}

Тестирование проводилось на ноутбуке, включенном в сеть электропитания. Во время тестирования ноутбук был нагружен только встроенными приложениями окружения, а также непосредственно системой тестирования.

\section{Демонстрация работы программы}

На рисунке \ref{img:primer} представлен результат работы программы.

\img{80mm}{primer}{Пример работы программы}

\section{Время выполнения алгоритмов}

Алгоритмы тестировались при помощи функции process\_time() из библиотеки time языка Python. Данная функция всегда возвращает значения времени, а имеено сумму системного и пользовательского процессорного времени текущего процессора, типа float в секундах.

Контрольная точка возвращаемого значения не определна, поэтому допустима только разница между результатами последовательных вызовов.

Результаты замеров приведены в таблицах \ref{tbl:best}, \ref{tbl:wor} и \ref{tbl:random}.

На рисунках \ref{img:1}, \ref{img:2} и \ref{img:3}, приведены графики зависимостей времени работы алгоритмов сортировки от размеров массивов на отсортированных, обратно отсортированных и случайных данных.


\img{100mm}{1}{Зависимость времени работы алгоритма сортировки от размера отсортированного массива (мск)}

\begin{table}[h]
	\begin{center}
		\caption{Время работы алгоритмов сортировки на отсортированных данных (мск)}
		\label{tbl:best}
		\begin{tabular}{|c|c|c|c|}
			\hline
			 Размер & Шейкером &  Вставками &  Выбором \\
			\hline
			100 & 5.2410 & 10.0906 & 258.0546\\
			\hline
			200 & 9.2891 & 23.5862 & 873.4441\\
			\hline
			300 & 15.6069 & 39.2295 & 1670.5604 \\
			\hline
			400 & 21.3040 & 51.9447 & 2904.2619 \\
			\hline
			500 & 27.7164 & 64.3753 & 4678.6021 \\
			\hline
			600 & 28.7513 & 72.2639 & 6565.3837 \\
			\hline
			1000 & 51.1101 & 96.3444 & 20440.3201 \\
			\hline
			2000 & 168.7951 & 327.707139 & 120891.3995 \\
			\hline
			2500 & 209.5140 & 394.2722 & 160569.3123 \\
			\hline
		\end{tabular}
	\end{center}
	
\end{table}
\clearpage


\img{100mm}{2}{Зависимость времени работы алгоритма сортировки от размера массива, отсортированного в обратном порядке (мск)}

\begin{table}[h]
	\begin{center}
		\caption{ Время работы алгоритмов сортировки на обратно	отсортированных данных (мск)}
		\label{tbl:wor}
		\begin{tabular}{|c|c|c|c|}
			\hline
			Размер & Шейкером &  Вставками &  Выбором \\
			\hline
			100 & 639.0698 & 363.9547 & 179.1664\\
			\hline
			200 & 2195.5410 & 1475.2803 & 735.1370 \\
			\hline
			300 & 4819.4516 & 3400.9440 & 1677.5576 \\
			\hline
			400 & 9230.5019 & 6217.0601 & 2988.1903 \\
			\hline
			500 & 18682.5808 & 15383.5356 & 8015.5767 \\
			\hline
			600 & 33239.3775 & 22488.2615 & 11754.5552 \\
			\hline
			1000 & 942220.648 & 64293.5541 & 33109.7468 \\
			\hline
			2000 & 370076.0900 & 255556.7835 & 131883.3811 \\
			\hline
			2500 & 597217.8193 & 372790.2408 & 188786.7314 \\
			\hline
		\end{tabular}
	\end{center}
	
\end{table}
\clearpage

\img{100mm}{3}{Зависимость времени работы алгоритма сортировки от размера массива, заполненного в случайном порядке (мск)}

\begin{table}[h]
	\begin{center}
		\caption{ Время работы алгоритмов сортировки на случайных данных (мск)}
		\label{tbl:random}
		\begin{tabular}{|c|c|c|c|}
			\hline
			Размер & Шейкером&  Вставками  &  Выбором \\
			\hline
			100 & 438.8057  & 186.5141 & 204.1938\\
			\hline
			200 & 1455.2835  & 692.1317& 725.6267 \\
			\hline
			300 & 3156.4853  & 1563.0922& 1667.3059 \\
			\hline
			400 & 5716.6090  & 4964.3020& 4954.9819 \\
			\hline
			500 & 14948.6332 & 7955.9345  & 8240.3997\\
			\hline
			600 & 21706.8825 & 11442.8451  & 11955.7538\\
			\hline
			1000 & 60019.2610  & 31373.1001 & 33769.7239\\
			\hline
			2000 & 250424.1895 & 130588.0997  & 141522.1266\\
			\hline
			2500 & 414524.0602 & 214781.0473  & 236668.4435\\
			\hline
		\end{tabular}
	\end{center}
	
\end{table}
\clearpage

\section*{Вывод}

Алгоритм сортировки вставками работает лучше остальных двух на случайных числах и уже отсортированных. Интерес представляет лишь первый случай, на котором сортировка вставками стабильно быстрее двух других алгоритмов.


