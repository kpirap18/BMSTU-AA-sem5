\chapter{Аналитическая часть}
В этом разделе будут представлены описания алгоритмов сортировки перемешиванием, вставками и выбором.

\section{Сортировка перемешиванием}

\textbf{Сортировка перемешиванием} \cite{sheyker} — это разновидность сортировки пузырьком. Отличие в том, что данная сортировка в рамках одной итерации проходит по массиву в обоих направлениях (слева направо и справа налево), тогда как сортировка пузырьком - только в одном направлении (слева направо).

Общие идеи алгоритма:
\begin{itemize}
	\item обход массива слева направо, аналогично пузырьковой - сравнение соседних элементов, меняя их местами, если левое значение больше правого;
	\item обход массива в обратном направлении (справа налево), начиная с элемента, который находится перед последним отсортированным, то есть на этом этапе элементы также сравниваются между собой и меняются местами, чтобы наименьшее значение всегда было слева.
\end{itemize}


\section{Сортировка вставками}

\textbf{Сортировка вставками \cite{insert}} — алгоритм сортировки, котором элементы входной последовательности просматриваются по одному, и каждый новый поступивший элемент размещается в подходящее место среди ранее упорядоченных элементов.

В начальный момент отсортированная последовательность пуста. На
каждом шаге алгоритма выбирается один из элементов входных данных и помешается на нужную позицию в уже отсортированной последовательности до тех пор, пока набор входных данных не будет исчерпан. В любой момент времени в отсортированной последовательности элементы удовлетворяют требованиям к входным данным алгоритма.

\section{Сортировка выбором}

\textbf{Сортировка выбором \cite{select}} - алгоритм, основанный на сравнение каждого элемента с каждым, и в случае необходимости производя обмен.

Шаги алгоритма.
\begin{enumerate}
	\item Находим номер минимального значения в текущем массиве;
	\item Производим обмен этого значения со значением первой неотсортированной позиции (обмен не нужен, если минимальный элемент уже находится на данной позиции);
	\item Далее сортируем ``хвост'' массива, исключив из рассмотрения уже отсортированные элементы.
\end{enumerate}

Для реализации устойчивости алгоритма необходимо в пункте 2 минимальный элемент непосредственно вставлять в первую неотсортированную позицию, не меняя порядок остальных элементов, что может привести к резкому увеличению числа обменов. 


\section*{Вывод}

В данной работе стоит задача реализации 3 алгоритмов сортировки, а
именно: перемешиванием, вставками и выбором. Необходимо оценить теоретическую оценку алгоритмов и проверить ее экспериментально.



