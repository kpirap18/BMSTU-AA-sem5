\chapter{Конструкторская часть}
В этом разделе будут приведены схемы алгоритмов и вычисления трудоемкости данных алгоритмов.

%\section{Требования к вводу}

%К вводу предъявляются следующие требования.
%\begin{itemize}
%	\item на вход программа получает выбранный пункт меню (цифра от 0 до 6);
%	\item на вход подается массив из целых чисел.
%\end{itemize}

%\section{Требования к программе}

%К программе продъявляются следующие требования.
%\begin{enumerate}
%	\item Две пустые строки - корректный ввод, программа не должна аварийно завершаться.
%	\item На выход программа должна вывести число - расстояние Левенштейна (Дамерау-Левенштейна), матрицу при необходимости.
%\end{enumerate}

\section{Разработка алгоритмов}

На рисунках  \ref{img:vyb}, \ref{img:perem} и \ref{img:vst} представлены схемы алгоритмов сортировки пузырьком, выбором и вставками соответственно.


\img{170mm}{vyb}{Схема алгоритма сортировки выбором}
\img{240mm}{perem}{Схема алгоритма сортировки перемешиванием}
\img{170mm}{vst}{Схема алгоритма сортировки вставками}

\section{Модель вычислений}


\section{Трудоёмкость алгоритм}


\subsection{Алгоритм сортировки перемешиванием}


\subsection{Алгоритм сортировки вставками}


\subsection{Алгоритм сортировки выбором}


\section*{Вывод}



