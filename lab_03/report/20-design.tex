\chapter{Конструкторская часть}
В этом разделе будут приведены схемы алгоритмов и вычисления трудоемкости данных алгоритмов.

%\section{Требования к вводу}

%К вводу предъявляются следующие требования.
%\begin{itemize}
%	\item на вход программа получает выбранный пункт меню (цифра от 0 до 6);
%	\item на вход подается массив из целых чисел.
%\end{itemize}

%\section{Требования к программе}

%К программе продъявляются следующие требования.
%\begin{enumerate}
%	\item Две пустые строки - корректный ввод, программа не должна аварийно завершаться.
%	\item На выход программа должна вывести число - расстояние Левенштейна (Дамерау-Левенштейна), матрицу при необходимости.
%\end{enumerate}

\section{Разработка алгоритмов}

На рисунках  \ref{img:vyb}, \ref{img:perem} и \ref{img:vst} представлены схемы алгоритмов сортировки пузырьком, выбором и вставками соответственно.


\img{170mm}{vyb}{Схема алгоритма сортировки выбором}
\img{240mm}{perem}{Схема алгоритма сортировки перемешиванием}
\img{170mm}{vst}{Схема алгоритма сортировки вставками}

\section{Модель вычислений (оценки трудоемкости)}

Для последующего вычисления трудоемкости необходимо ввести модель вычислений:
\begin{enumerate}
	\item операции из списка (\ref{for:opers}) имеют трудоемкость 1;
	\begin{equation}
		\label{for:opers}
		+, -, /, *, \%, =, +=, -=, *=, /=, \%=, ==, !=, <, >, <=, >=, [], ++, {-}-
	\end{equation}
	\item трудоемкость оператора выбора \code{if условие then A else B} рассчитывается, как (\ref{for:if});
	\begin{equation}
		\label{for:if}
		f_{if} = f_{\text{условия}} +
		\begin{cases}
			f_A, & \text{если условие выполняется,}\\
			f_B, & \text{иначе.}
		\end{cases}
	\end{equation}
	\item трудоемкость цикла рассчитывается, как (\ref{for:for});
	\begin{equation}
		\label{for:for}
		f_{for} = f_{\text{инициализации}} + f_{\text{сравнения}} + N(f_{\text{тела}} + f_{\text{инкремент}} + f_{\text{сравнения}})
	\end{equation}
	\item трудоемкость вызова функции равна 0.
\end{enumerate}

\section{Трудоёмкость алгоритм}

Обозначим во всех последующих вычислениях размер массивов как N.

\subsection{Алгоритм сортировки перемешиванием}

\begin{itemize}
	\item Трудоёмкость сравнения внешнего цикла $WHILE(swap == True)$, которая равна (\ref{for:bubble_outer}):
	\begin{equation}
		\label{for:bubble_outer}
		f_{outer} = 1 + 2 \cdot (N - 1)
	\end{equation}
	\item Суммарная трудоёмкость внутренних циклов, количество итераций которых меняется в промежутке $[1..N-1]$, которая равна (\ref{for:bubble_inner}):
	\begin{equation}
		\label{for:bubble_inner}
		f_{inner} = 5(N - 1) + \frac{2 \cdot (N - 1)}{2} \cdot (3 + f_{if})
	\end{equation}
	\item Трудоёмкость условия во внутреннем цикле, которая равна (\ref{for:bubble_if}):
	\begin{equation}
		\label{for:bubble_if}
		f_{if} = 4 + \begin{cases}
			0, & \text{л.с.}\\
			9, & \text{х.с.}\\
		\end{cases}
	\end{equation}
\end{itemize}

Трудоёмкость в лучшем случае (\ref{for:bubble_best}):
\begin{equation}
	\label{for:bubble_best}
	f_{best} = -3 + \frac{3}{2} N + \approx \frac{3}{2} N = O(N)
\end{equation}

Трудоёмкость в худшем случае (\ref{for:bubble_worst}):
\begin{equation}
	\label{for:bubble_worst}
	f_{worst} = -3 - 8N + 8N^2 \approx 8N^2 = O(N^2)
\end{equation}



\subsection{Алгоритм сортировки выбором}


Трудоёмкость алгоритма сортировки выбором состоит из:
\begin{itemize}
	\item Трудоёмкость сравнения, инкремента внешнего цикла, а также зависимых только от него операций, по $i \in [1..N)$, которая равна (\ref{for:selection_outer}):
	\begin{equation}
		\label{for:selection_outer}
		f_{outer} = 2 + 12(N - 1)
	\end{equation}
	\item Суммарная трудоёмкость внутренних циклов, количество итераций которых меняется в промежутке $[1..N-1]$, которая равна (\ref{for:selection_inner}):
	\begin{equation}
		\label{for:selection_inner}
		f_{inner} = 2(N - 1) + \frac{N \cdot (N - 1)}{2} \cdot f_{if}
	\end{equation}
	\item Трудоёмкость условия во внутреннем цикле, которая равна (\ref{for:selection_if}):
	\begin{equation}
		\label{for:selection_if}
		f_{if} = 3 + \begin{cases}
			0, & \text{л.с.}\\
			3, & \text{х.с.}\\
		\end{cases}
	\end{equation}
\end{itemize}

Трудоёмкость в лучшем случае (\ref{for:selection_best}):
\begin{equation}
	\label{for:selection_best}
	f_{best} = -12 + 12.5N + \frac{3}{2}N^2 \approx \frac{3}{2}N^2 = O(N^2)
\end{equation}

Трудоёмкость в худшем случае (\ref{for:selection_worst}):
\begin{equation}
	\label{for:selection_worst}
	f_{worst} = -12 + 11N + 3N^2 \approx 3N^2 = O(N^2)
\end{equation}





\subsection{Алгоритм сортировки вставками}

Трудоёмкость сортировки вставками может быть рассчитана таким же образом, что и трудоёмкости алгоритмов выше. Асимптотическая трудоёмкость алгоритма сортировки вставками $O(N^2)$ в худшем случае и $O(N)$ --- в лучшем \cite{insert}.





\section*{Вывод}

Были разработаны схемы всех трех алгоритмов сортировки. Для каждого из них были рассчитаны и оценены лучшие и худшие случаи.


