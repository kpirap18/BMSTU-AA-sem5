\chapter{Исследовательская часть}

В данном разделе будут приведены примеры работы программ, постановка эксперимента и сравнительный анализ алгоритмов на основе полученных данных.

\section{Технические характеристики}

Технические характеристики устройства, на котором выполнялось тестирование:

\begin{itemize}
	\item операционная система: Ubuntu 20.04.3 \cite{ubuntu} Linux \cite{linux} x86\_64;
	\item память: 8 GiB;
	\item процессор: 11th Gen Intel® Core™ i5-1135G7 @ 2.40GHz \cite{intel}.
\end{itemize}

Тестирование проводилось на ноутбуке, включенном в сеть электропитания. Во время тестирования ноутбук был нагружен только встроенными приложениями окружения, а также непосредственно системой тестирования.

\section{Демонстрация работы программы}

На рисунке \ref{img:standart} представлен результат работы программы.

\clearpage

\img{145mm}{example}{Пример работы программы}


\section{Время выполнения алгоритмов}

Результаты замеров приведены в таблицах \ref{tab:time1} и \ref{tab:time2}. Некоторые обозначения в таблице: СА - стандартный алгоритм, АВ - алгоритм Винограда, ОАВ - оптимизированный алгоритм Винограда.

На рисунках \ref{img:1} и \ref{img:2} приведены графики зависимостей времени работы алгоритмов от размеров матриц.

Время в таблицах и на рисунках приведены в секундах.

\begin{table}[h]
	\begin{center}
		\caption{\label{tab:time1}Результаты замеров времени алгоритмов при четных размерах матриц (сек)}
		\begin{tabular}{|c|c|c|c|c|}
		\hline
		Размер & АВ &  СА & ОАВ \\
		\hline
		50  & 0.0138 & 0.0122 & 0.0108\\
		\hline
		100  & 0.1054 & 0.1001 & 0.0869\\
		\hline
		150  & 0.5808 & 0.4479 & 0.3051 \\
		\hline
		200  & 1.3738 & 0.8694 & 1.1329 \\
		\hline
		300  & 4.3151 & 4.4543 & 3.3223 \\
		\hline
		400  & 11.4547 & 10.6556 & 9.6063 \\
		\hline
		500  & 18.8731 & 16.9742 & 14.1837 \\
		\hline
		
		\end{tabular}
	\end{center}
\end{table}

\img{130mm}{1}{Зависимость времени работы алгоритма от четного размера квадратной матрицы (сек)}

\begin{table}[h]
	\begin{center}
		\caption{\label{tab:time2}Результаты замеров времени алгоритмов при нечетных размерах матриц (сек)}
		\begin{tabular}{|c|c|c|c|c|}
		\hline
		Размер & АВ &  СА & ОАВ \\
		\hline
		50  &0.0294 & 0.0237 & 0.0273\\
		\hline
		100  &0.2271 & 0.1751 & 0.1681\\
		\hline
		150  &0.7349 & 0.5922 & 0.5510 \\
		\hline
		200  &1.8234 &1.4086 & 1.3162 \\
		\hline
		300  & 6.4540 &4.7933 & 4.6446 \\
		\hline
		400  & 14.8665 & 11.6306 & 11.3941 \\
		\hline
		500  & 28.6478 & 22.6672 & 21.6239 \\
		\hline
		\end{tabular}
	\end{center}
\end{table}

\img{130mm}{2}{Зависимость времени работы алгоритма от нечетного размера квадратной матрицы (сек)}

\clearpage

\section*{Вывод}

В результате проведенного эксперимента был получен следующий вывод: алгоритм Винограда работает дольше за счет большого количества операций. Оптимизированный алгоритм работает быстрее, так как там меньше операций по сравнению с обычным алгоритмом Винограда и стандартным алгоритмом.  