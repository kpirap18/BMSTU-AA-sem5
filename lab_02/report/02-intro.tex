\chapter*{Введение}
\addcontentsline{toc}{chapter}{Введение}

Термин «матрица» применяется во множестве разных областей: от
программирования до кинематографии.

Матрица в математике – это таблица чисел, состоящая из определенного количества строк (m) и столбцов (n).

Мы встречаемся с матрицами каждый день, так как любая числовая информация, занесенная в таблицу, уже в какой-то степени считается матрицей.


Целью работы работы является изучение и реализация алгоритмов
умножения матриц, вычисление трудоёмкости этих алгоритмов. В данной
лабораторной работе рассматривается стандартный алгоритм умножения
матриц, алгоритм Винограда и модифицированный алгоритм Винограда.


Для достижения цели ставятся следующие задачи.


\begin{enumerate}
	\item Изучить классический алгоритм умножения матриц, алгоритм Винограда и модифицированный алгоритм Винограда.
	\item Реализовать классический алгоритм умножения матриц, алгоритм
	Винограда и модифицированный алгоритм Винограда.

	\item Дать оценку трудоёмкости алгоритмов.
	\item Замерить время работы алгоритмов.
	\item Описать и обосновать полученные результаты в отчете о выполненной лабораторной работе, выполненном как расчётно-пояснительная
	записка к работе.
\end{enumerate}
