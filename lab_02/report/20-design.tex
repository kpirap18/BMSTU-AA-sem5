\chapter{Конструкторская часть}
В этом разделе будут приведены требования к вводу и программе, а также схемы алгоритмов умножения матриц.

\section{Разработка алгоритмов}

Предполагается, что на вход всех алгоритмов поступили матрицы верного размера.

На рисунке \ref{img:standart} приведена схема стандартного алгоритма умножения матриц.

\img{150mm}{standart}{Схема стандартного алгоритма умножения матриц}
\clearpage

На рисунках \ref{img:vinograd1} и \ref{img:vinograd2} приведена схема алгоритма Винограда умножения матриц.

\img{150mm}{vinograd1}{Схема алгоритма Винограда умножения матриц}
\clearpage
\img{150mm}{vinograd2}{Схема алгоритма Винограда умножения матриц(продолжение)}
\clearpage

На рисунках \ref{img:vinograd1_op} и \ref{img:vinograd2_op} приведена схема оптимизировванного алгоритма Винограда умножения матриц.

\img{150mm}{vinograd1_op}{Схема оптимизированного алгоритма Винограда умножения матриц}
\clearpage
\img{150mm}{vinograd2_op}{Схема оптимизированного алгоритма Винограда умножения матриц(продолжение)}
\clearpage


\section{Модель вычислений}

Для последующего вычисления трудоемкости необходимо ввести модель вычислений:
\begin{enumerate}
	\item операции из списка (\ref{for:opers}) имеют трудоемкость 1;
	\begin{equation}
		\label{for:opers}
		+, -, *, /, \%, ==, !=, <, >, <=, >=, [], ++, {-}-
	\end{equation}
	\item трудоемкость оператора выбора \code{if условие then A else B} рассчитывается, как (\ref{for:if});
	\begin{equation}
		\label{for:if}
		f_{if} = f_{\text{условия}} +
		\begin{cases}
			f_A, & \text{если условие выполняется,}\\
			f_B, & \text{иначе.}
		\end{cases}
	\end{equation}
	\item трудоемкость цикла рассчитывается, как (\ref{for:for});
	\begin{equation}
		\label{for:for}
		f_{for} = f_{\text{инициализации}} + f_{\text{сравнения}} + N(f_{\text{тела}} + f_{\text{инкремента}} + f_{\text{сравнения}})
	\end{equation}
	\item трудоемкость вызова функции равна 0.
\end{enumerate}


\section{Трудоемкость алгоритмов}

В следующих частях будут расчитаны трудоемкости алгоритмов умножения матриц.

\subsection{Стандартный алгоритм умножения матриц}

Во всех последующих алгоритмах не будем учитывать инициализацию матрицы, в которую записывается результат, потому что данное действие есть во всех алгоритмах и при этом не является самым трудоёмким.

Трудоёмкость стандартного алгоритма умножения матриц состоит из:
\begin{itemize}
	\item внешнего цикла по $i \in [1..M]$, трудоёмкость которого: $f = 2 + M \cdot (2 + f_{body})$;
	\item цикла по $j \in [1..N]$, трудоёмкость которого: $f = 2 + N \cdot (2 + f_{body})$;
	\item цикла по $k \in [1..K]$, трудоёмкость которого: $f = 2 + 10K$;
\end{itemize}

Учитывая, что трудоёмкость стандартного алгоритма равна трудоёмкости внешнего цикла, можно вычислить ее, подставив циклы тела (\ref{for:standard}):
\begin{equation}
	\label{for:standard}
	f_{standard} = 2 + M \cdot (4 + N \cdot (4 + 10K)) = 2 + 4M + 4MN + 10MNK \approx 10MNK
\end{equation}

\subsection{Алгоритм Копперсмита — Винограда}

Трудоёмкость алгоритма Копперсмита — Винограда состоит из:
\begin{itemize}
	\item создания и инициализации массивов MH и MV, трудоёмкость которого (\ref{for:init}):
	\begin{equation}
		\label{for:init}
		f_{init} = M + N;
	\end{equation}
	
	\item заполнения массива MH, трудоёмкость которого (\ref{for:MH}):
	\begin{equation}
		\label{for:MH}
		f_{MH} = 2 + K (2 + \frac{M}{2} \cdot 11);
	\end{equation}
	
	\item заполнения массива MV, трудоёмкость которого (\ref{for:MV}):
	\begin{equation}
		\label{for:MV}
		f_{MV} = 2 + K (2 + \frac{N}{2} \cdot 11);
	\end{equation}
	
	\item цикла заполнения для чётных размеров, трудоёмкость которого (\ref{for:cycle}):
	\begin{equation}
		\label{for:cycle}
		f_{cycle} = 2 + M \cdot (4 + N \cdot (11 + \frac{K}{2} \cdot 23));
	\end{equation}
	
	\item цикла, для дополнения умножения суммой последних нечётных строки и столбца, если общий размер нечётный, трудоемкость которого (\ref{for:last}):
	\begin{equation}
		\label{for:last}
		f_{last} = \begin{cases}
			2, & \text{чётная,}\\
			4 + M \cdot (4 + 14N), & \text{иначе.}
		\end{cases}
	\end{equation}
\end{itemize}

Итого, для худшего случая (нечётный общий размер матриц) имеем (\ref{for:bad}):
\begin{equation}
	\label{for:bad}
	f =  f_{MH} + f_{MV} + f_{cycle} + f_{last}\approx 11.5 \cdot MNK
\end{equation}

Для лучшего случая (чётный общий размер матриц) имеем (\ref{for:good}):
\begin{equation}
	\label{for:good}
f =  f_{MH} + f_{MV} + f_{cycle} + f_{last} \approx 11.5 \cdot MNK
\end{equation}


\subsection{Оптимизированный алгоритм Копперсмита — Винограда}

Оптимизированный алгоритм Винограда представляет собой обычный алгоритм Винограда, за исключением следующих оптимизаций:
\begin{itemize}
	\item вычисление происходит заранее;
	\item используется битовый сдвиг, вместо деления на 2;
	\item последний цикл для нечётных элементов включён в основной цикл,
	используя дополнительные операции в случае нечётности N.
\end{itemize}


Трудоёмкость улучшенного алгоритма Копперсмита — Винограда состоит из:
\begin{itemize}
	\item создания и инициализации массивов MH и MV, трудоёмкость которого (\ref{for:impr_init}):
	\begin{equation}
		\label{for:impr_init}
		f_{init} = M + N;
	\end{equation}
	
	\item заполнения массива MH, трудоёмкость которого (\ref{for:impr_MH}):
	\begin{equation}
		\label{for:impr_MH}
		f_{MH} =  2 + K (2 + \frac{M}{2} \cdot 8);
	\end{equation}
	
	\item заполнения массива MV, трудоёмкость которого (\ref{for:impr_MV}):
	\begin{equation}
		\label{for:impr_MV}
		f_{MV} =  2 + K (2 + \frac{M}{2} \cdot 8);
	\end{equation}
	
	\item цикла заполнения для чётных размеров, трудоёмкость которого (\ref{for:impr_cycle}):
	\begin{equation}
		\label{for:impr_cycle}
		f_{cycle} =2 + M \cdot (4 + N \cdot (11 + \frac{K}{2} \cdot 18));
	\end{equation}
	
	\item условие, для дополнения умножения суммой последних нечётных строки и столбца, если общий размер нечётный, трудоемкость которого (\ref{for:impr_last}):
	\begin{equation}
		\label{for:impr_last}
		f_{last} = 
		\begin{cases}
			1, & \text{чётная,}\\
			4 + M \cdot (4 + 10N), & \text{иначе.}
		\end{cases}
	\end{equation}
\end{itemize}

Итого, для худшего случая (нечётный общий размер матриц) имеем (\ref{for:bad_impr}):
\begin{equation}
	\label{for:bad_impr}
	f = f_{MH} + f_{MV} + f_{cycle} + f_{last} \approx 9MNK
\end{equation}

Для лучшего случая (чётный общий размер матриц) имеем (\ref{for:good_impr}):
\begin{equation}
	\label{for:good_impr}
	f = f_{MH} + f_{MV} + f_{cycle} + f_{last} \approx 9MNK
\end{equation}


\section*{Вывод}

На основе теоретических данных, полученных из аналитического раздела, были построены схемы обоих алгоритмов умножения матриц.  Оценены их трудоёмкости в лучшем и худшем случаях.
