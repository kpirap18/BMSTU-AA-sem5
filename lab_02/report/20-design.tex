\chapter{Конструкторская часть}
В этом разделе будут приведены требования к вводу и программе, а также схемы алгоритмов умножения матриц.

\section{Разработка алгоритмов}

На рисунке \ref{img:standart} приведена схема стандартного алгоритма умножения матриц.

На рисунках \ref{img:vinograd1} и \ref{img:vinograd2} приведена схема алгоритма Винограда умножения матриц.




\img{150mm}{standart}{Схема стандартного алгоритма умножения матриц}
\clearpage
\img{150mm}{vinograd1}{Схема алгоритма Винограда умножения матриц}
\clearpage
\img{150mm}{vinograd2}{Схема алгоритма Винограда умножения матриц(продолжение)}
\clearpage



\section{Модель вычислений}

Для последующего вычисления трудоемкости необходимо ввести модель вычислений:
\begin{enumerate}
	\item операции из списка (\ref{for:opers}) имеют трудоемкость 1;
	\begin{equation}
		\label{for:opers}
		+, -, /, \%, ==, !=, <, >, <=, >=, [], ++, {-}-
	\end{equation}
	\item трудоемкость оператора выбора \code{if условие then A else B} рассчитывается, как (\ref{for:if});
	\begin{equation}
		\label{for:if}
		f_{if} = f_{\text{условия}} +
		\begin{cases}
			f_A, & \text{если условие выполняется,}\\
			f_B, & \text{иначе.}
		\end{cases}
	\end{equation}
	\item трудоемкость цикла рассчитывается, как (\ref{for:for});
	\begin{equation}
		\label{for:for}
		f_{for} = f_{\text{инициализации}} + f_{\text{сравнения}} + N(f_{\text{тела}} + f_{\text{инкремента}} + f_{\text{сравнения}})
	\end{equation}
	\item трудоемкость вызова функции равна 0.
\end{enumerate}


\section{Трудоемкость алгоритмов}

\subsection{Стандартный алгоритм умножения матриц}

\subsection{Алгоритм Копперсмита — Винограда}

\subsection{Оптимизированный алгоритм Копперсмита —
	Винограда}

\section*{Вывод}


