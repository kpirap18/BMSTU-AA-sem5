\chapter{Технологическая часть}
В данном разделе будут приведены требования к программному обеспечению, средства реализации и листинги кода.

\section{Выбор языка программирования}

В данной лабораторной работе использовался язык программирования - С\# \cite{csharplang}.

Данный язык является нативным.

В качестве среды разработки выбор сделан в сторону Visual Studio 2019 \cite{wind}. 

\section{Сведения о модулях программы}
Программа состоит из двух модулей:
\begin{enumerate}
	\item Program.cs - главный файл программы, в котором содержится точка входа в программу.
	\item Sort.cs - файл с реализацией алгоритма сортировки в двух варинатах (обычный и многопоточный).
\end{enumerate}

\section{Листинг кода}
В листингах \ref{lst:simple}, \ref{lst:mnog} представлены реализации алгоритма ранговой сортировки многопоточной и обычной.

Также на листинге \ref{lst:dop} представлен вариант реализации многопоточной программы для ранговой сортировки для замеров на больших массивов при 4 потоках.

\begin{lstlisting}[label=lst:simple,caption=Алгоритм ранговой сортировки (обычный)]
public static string Sequential(ref int[] a)
{
	int n = a.Length;
	int[] b = new int[n];
	int x;

	for (int i = 0; i < n; i++)
	{
		x = 0;
		for (int j = 0; j < n; j++)
			if (a[i] > a[j] || (a[i] == a[j] && j > i))
				x++;
		b[x] = a[i]; 
	}
	return str;
}
\end{lstlisting}


\begin{lstlisting}[label=lst:mnog,caption=Алгоритм ранговой сортировки (многопоточный)]
public static string Parallel(int[] a)
{
	int M = 4;
	Thread[] thrs = new Thread[M];
	int n = a.Length;

	for (int i = 0; i < M; i++)
	{
		int thr = i;
		thrs[i] = new Thread(() =>
		{
			for (int j = thr; j < n; j += M)
			{
				int x = 0;
				for (int k = 0; k < n; k++)
					if (a[j] > a[k] || (a[j] == a[k] && k > j))
						x++;
				b[x] = a[j];
			}
		});
		thrs[i].Start();
	}
	for (int i = 0; i < M; i++) 
		thrs[i].Join();
	return str;
}
\end{lstlisting}

\begin{lstlisting}[label=lst:dop,caption=Алгоритм ранговой сортировки (обычный)]
public static int[] SortParall(this int[] a)
{
	int n = a.Length;
	int[] b = new int[n];
	Parallel.ForEach(Partitioner.Create(0, n), range =>
	{
		int x;
		for (int i = range.Item1; i < range.Item2; i++)
		{
			x = 0;
			
			for (int j = 0; j < n; j++)
				if (a[i] > a[j] || (a[i] == a[j] && j > i))
					x++;
			b[x] = a[i]; 
		}
	});
	return b;
}
\end{lstlisting}

\section{Тестирование}

В таблице \ref{tbl:functional_test} приведены тесты для функций, реализующих алгоритмы сортировки. Тесты пройдены успешно.


\begin{table}[h]
	\begin{center}
		\caption{\label{tbl:functional_test} Функциональные тесты}
		\begin{tabular}{|c|c|c|}
			\hline
			Входной массив & Ожидаемый результат & Результат \\ 
			\hline
			$[1,2,3,4]$ & $[1,2,3,4]$  & $[1,2,3,4]$\\
			$[5,4,3,2,1]$  & $[1,2,3,4,5]$ & $[1,2,3,4,5]$\\
			$[3,2,-5,0,1]$  & $[-5,0,0,2,3]$  & $[-5,0,0,2,3]$\\
			$[4]$  & $[4]$  & $[4]$\\
			$[]$  & $[]$  & $[]$\\
			\hline
		\end{tabular}
	\end{center}
\end{table}


\section*{Вывод}
В данном разделе были разобраны листинги  показывающие
работу как однопоточного, так и многопоточного алгоритма ранговой сортировки.
