\chapter{Исследовательская часть}
В данном разделе будет произведено сравнение вышеизложенного алгоритма (однопоточная и многопоточная реализация).

\section{Технические характеристики}

Технические характеристики устройства, на котором выполнялось тестирование, следующие.

\begin{itemize}
	\item Операционная система: Windows 10 \cite{oswind} x86\_64.
	\item Память: 8 GiB.
	\item Процессор: 11th Gen Intel® Core™ i5-1135G7 @ 2.40GHz \cite{intel}.
	\item 4 физических ядра и 8 логических ядра.
\end{itemize}

Тестирование проводилось на ноутбуке, включенном в сеть электропитания. Во время тестирования ноутбук был нагружен только встроенными приложениями окружения, а также непосредственно системой тестирования.

\section{Время выполнения алгоритмов}

Результаты замеров приведены в таблицах \ref{tbl:allpotok} и \ref{tbl:only4}.
На рисунках \ref{img:f111}, \ref{img:f222} и \ref{img:f333} приведены графики зависимостей времени работы реализации алгоритма ранговой сортировки от размеров массивов для различного количества потоков. 


\begin{table}[h]
		\caption{Время (мс) работы реализации алгоритма ранговой сортировки}
	\label{tbl:allpotok}
	\begin{center}
		\begin{tabular}{|c|c|c|c|c|c|}
			\hline
			& \multicolumn{3}{c|}{\bfseries Размер массива}           \\ \cline{2-4}
			\bfseries Кол-во потоков & \bfseries 10 000 & \bfseries 50 000 & \bfseries 100 000
			\csvreader{inc/csv/random2.csv}{}
			{\\\hline \csvcoli&\csvcolii&\csvcoliii&\csvcoliv}
			\\\hline
		\end{tabular}
	\end{center}
\end{table}


\begin{table}[h]
	\caption{Время (мс) работы реализации алгоритма ранговой сортировки при оптимальном количестве потоков (такое количество выбирает сам метод)}
	\label{tbl:only4}
	\begin{center}
		\begin{tabular}{|c|c|c|c|c|}
			\hline
			& \multicolumn{2}{c|}{\bfseries Тип алгоритма}           \\ \cline{2-3}
			\bfseries Размер & \bfseries Обычный & \bfseries Многопоточный 
			\csvreader{inc/csv/only.csv}{}
			{\\\hline \csvcoli&\csvcolii&\csvcoliii}
			\\\hline
		\end{tabular}
	\end{center}
\end{table}

\clearpage

\img{100mm}{f111}{Зависимость времени от кол-ва потока для массива размера 10 000 элементов}

\img{100mm}{f222}{Зависимость времени от кол-ва потока для массива размера 50 000 элементов}

\img{100mm}{f333}{Зависимость времени от кол-ва потока для массива размера 100 000 элементов}

\img{120mm}{f4}{Зависимость времени от длины массива при оптимальном количестве потоков (такое количество выбирает сам метод)}
\clearpage



\section*{Вывод}

В данном разделе было произведено сравнение алгоритма ранговой сортировки при однопоточной реализации и многопоточной. Результат показал, что выгоднее всего использовать 16 потоков (можно увидеть, что именно при 16 потоков реализация алгоритма работает быстрее).  