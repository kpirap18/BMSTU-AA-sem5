\chapter*{Введение}
\addcontentsline{toc}{chapter}{Введение}

Многопоточность — способность центрального процессора (CPU) или одного ядра в многоядерном процессоре одновременно выполнять несколько процессов или потоков, соответствующим образом поддерживаемых операционной системой.
Этот подход отличается от многопроцессорности, так как многопоточность процессов и потоков совместно использует ресурсы одного или нескольких ядер: вычислительных блоков, кэш-памяти ЦПУ или буфера перевода с преобразованием (TLB).

В тех случаях, когда многопроцессорные системы включают в себя несколько полных блоков обработки, многопоточность направлена на максимизацию использования ресурсов одного ядра, используя параллелизм на уровне потоков, а также на уровне инструкций.
Поскольку эти два метода являются взаимодополняющими, их иногда объединяют в системах с несколькими многопоточными ЦП и в ЦП с несколькими многопоточными ядрами.

Многопоточная парадигма стала более популярной с конца 1990-х годов, поскольку усилия по дальнейшему использованию параллелизма на уровне инструкций застопорились.
Смысл многопоточности — квазимногозадачность на уровне одного исполняемого процесса.
Значит, все потоки процесса помимо общего адресного пространства имеют и общие дескрипторы файлов. Выполняющийся процесс имеет как минимум один (главный) поток.

Многопоточность (как доктрину программирования) не следует путать ни с многозадачностью, ни с многопроцессорностью, несмотря на то, что операционные системы, реализующие многозадачность, как правило, реализуют и многопоточность.

Достоинства:
\begin{itemize}
	\item облегчение программы посредством использования общего адресного пространства;
	\item меньшие затраты на создание потока в сравнении с процессами;
	\item повышение производительности процесса за счёт распараллеливания процессорных вычислений;
	\item если поток часто теряет кэш, другие потоки могут продолжать использовать неиспользованные вычислительные ресурсы.
\end{itemize}

Недостатки:
\begin{itemize}
	\item несколько потоков могут вмешиваться друг в друга при совместном использовании аппаратных ресурсов \cite{Nemirovsky};
	\item с программной точки зрения аппаратная поддержка многопоточности более трудоемка для программного обеспечения \cite{Olukotun};
	\item проблема планирования потоков;
	\item специфика использования. 
\end{itemize}

Вручную настроенные программы на ассемблере, использующие расширения MMX или AltiVec и выполняющие предварительные выборки данных, не страдают от потерь кэша или неиспользуемых вычислительных ресурсов. Таким образом, такие программы не выигрывают от аппаратной многопоточности и действительно могут видеть ухудшенную производительность из-за конкуренции за общие ресурсы.

Однако несмотря на количество недостатков, перечисленных выше, многопоточная парадигма имеет большой потенциал на сегодняшний день и при должном написании кода позволяет значительно ускорить однопоточные алгоритмы.


В рамках выполнения работы необходимо решить следующие задачи.


\begin{enumerate}
	\item Изучения основ параллельных вычислений
	\item Применение изученных основ для реализации многопоточной сортировки.

	\item Получения практических навыков.
	\item Произвести сравнительный анализ параллельной и однопоточной реализации алгоритма сортировки.
	\item Экспериментальное подтверждение различий во временной эффективности реализации однопоточной и многопоточной ранговой сортировки.
	\item Выбор и обоснование языка программирования, для решения данной задачи.
	\item Описание и обоснование полученных результатов в отчете о выполненной лабораторной работе, выполненного как расчётно-пояснительная записка к работе.
\end{enumerate}
