\chapter{Исследовательская часть}
В данном разделе будет произведено сравнение вышеизложенного алгоритма (однопоточная и многопоточная реализация).

\section{Технические характеристики}

Технические характеристики устройства, на котором выполнялось тестирование, следующие.

\begin{itemize}
	\item Операционная система: Windows 10 \cite{oswind} x86\_64.
	\item Память: 8 GiB.
	\item Процессор: 11th Gen Intel® Core™ i5-1135G7 @ 2.40GHz \cite{intel}.
	\item 4 физических ядра и 8 логических ядра.
\end{itemize}

Тестирование проводилось на ноутбуке, включенном в сеть электропитания. Во время тестирования ноутбук был нагружен только встроенными приложениями окружения, а также непосредственно системой тестирования.

\section{Пример выполнения программы}
На рисунке \ref{img:prim} представлен результат работы программы.

\img{190mm}{prim}{Пример работы программы}

\clearpage

\section{Время выполнения алгоритмов}

Для проведения временного анализа, программа запускалась для 10 задач.

Результаты сравнения времени работы алгоритмов последовательного и конвейерного, приведены в таблице \ref{tbl:only4}.


\begin{table}[h]
	\caption{Время (такт) выполнения параллельного и последовательного 		конвейеров в зависимости от длины очереди}
	\label{tbl:only4}
	\begin{center}
		\begin{tabular}{|c|c|c|c|c|}
			\hline
			& \multicolumn{2}{c|}{\bfseries Тип алгоритма}           \\ \cline{2-3}
			\bfseries Размер & \bfseries Последовательный & \bfseries Конвейерный 
			\csvreader{inc/csv/only1.csv}{}
			{\\\hline \csvcoli&\csvcolii&\csvcoliii}
			\\\hline
		\end{tabular}
	\end{center}
\end{table}

На рисунках \ref{img:f1} и \ref{img:f2} представлены графики зависимости времени от размера массива для последовательного и конвейерного алгоритма.

\img{100mm}{f1}{График зависимости времени (такт) от размера массива}

\img{100mm}{f2}{График зависимости времени (такт) от размера массива}

\clearpage


В таблице \ref{tbl:allpotok} представлены времена нахождения в очереди на определенную ленту конвейера. 

\begin{table}[h]
	\caption{Время (такт) нахождения в очереди на определенную ленту конвейера}
	\label{tbl:allpotok}
	\begin{center}
		\begin{tabular}{|c|c|c|c|c|c|}
			\hline
			& \multicolumn{3}{c|}{\bfseries Номер ленты}           \\ \cline{2-4}
			\bfseries Номер задачи & \bfseries 1 & \bfseries 2 & \bfseries 3
			\csvreader{inc/csv/random2.csv}{}
			{\\\hline \csvcoli&\csvcolii&\csvcoliii&\csvcoliv}
			\\\hline
		\end{tabular}
	\end{center}
\end{table}

Как можно заметить, время нахождения в очереди на ленте 1 самое большое, потому что все задачи сначала попадают в очередь ленты 1, и начинаются обрабатываться не сразу. 

В таблице \ref{tbl:ob} представлены времена обработки задачи на каждой ленте конвейера. 

\clearpage
\begin{table}[h]
	\caption{Время (такт) обработки задачи на каждой ленте конвейера}
	\label{tbl:ob}
	\begin{center}
		\begin{tabular}{|c|c|c|c|c|c|}
			\hline
			& \multicolumn{3}{c|}{\bfseries Номер ленты}           \\ \cline{2-4}
			\bfseries Номер задачи & \bfseries 1 & \bfseries 2 & \bfseries 3
			\csvreader{inc/csv/random.csv}{}
			{\\\hline \csvcoli&\csvcolii&\csvcoliii&\csvcoliv}
			\\\hline
		\end{tabular}
	\end{center}
\end{table}

Как можно заметить, лента 3 обработывает задачи быстрее первых двух лент. Объяснить это можно тем фактом, что трудоемкость алгоритма тертьей ленты зависит от числа, поступающего на вход данного алгоритма.


\section*{Вывод}

В данном разделе было произведено сравнение последовательной реализации трех алгоритмов и конвейера с использованием многопоточности. По результатам исследования можно сказать, что конвейерную обработку выгоднее применять на больших числах (большие длины массивов, большое количество задач), так как на малых размерах последовательный алгоритм выигрывает у конвейерного.