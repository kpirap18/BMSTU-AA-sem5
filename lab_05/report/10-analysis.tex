\chapter{Аналитическая часть}
В данном разделе представлены теоретические сведения о рассматриваемых алгоритмах.

\section{Описание конвейерной обработки данных}
Конвейер \cite{konveer} — способ организации вычислений, используемый в современных процессорах и контроллерах с целью повышения их производительности (увеличения числа инструкций, выполняемых в единицу времени — эксплуатация параллелизма на уровне инструкций), технология, используемая при разработке компьютеров и других цифровых электронных устройств.

Конвейерная обработка в общем случае основана на разделении подлежащей исполнению функции на более мелкие части, называемые лентами, и выделении для каждой из них отдельного блока аппаратуры. Так, обработку любой машинной команды можно разделить на несколько этапов (лент), организовав передачу данных от одного этапа к следующему.

Конвейерную обработку можно использовать для совмещения этапов выполнения разных команд. Производительность при этом возрастает благодаря тому, что одновременно на различных ступенях конвейера выполняются несколько команд.


\section{Описание алгоритмов}
В данной лабораторной работе предполагается создание программы на основе конвейрной обработки данных. В качестве алгоритмов на каждую из трех лент были выбраны следующие алгоритмы.
\begin{enumerate}
	\item Поиск среднего арифмитического значения в массиве.
	\item Подсчет количества числе в массиве больших, чем среднее значение.
	\item Определение того факта, является ли найденное количество простым числом или нет.
\end{enumerate}


\section{Вывод}
В данном разделе были рассмотрены основополагающие материалы, которые в дальнейшем потребуются для реализации алгоритмов. Были рассмотрены особенности построения конвейерных вычислений.

