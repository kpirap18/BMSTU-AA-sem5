\chapter{Аналитическая часть}
В данном разделе представлены теоретические сведения о рассматриваемых алгоритмах.

\section{Описание конвейреной обработки данных}
Конвейер \cite{konveer}— способ организации вычислений, используемый в современных процессорах и контроллерах с целью повышения их производительности (увеличения числа инструкций, выполняемых в единицу времени — эксплуатация параллелизма на уровне инструкций), технология, используемая при разработке компьютеров и других цифровых электронных устройств.

Идея заключается в параллельном выполнении нескольких инструкций процессора. Сложные инструкции процессора представляются в виде последовательности более простых стадий. Вместо выполнения инструкций последовательно (ожидания завершения конца одной инструкции и перехода к следующей), следующая инструкция может выполняться через несколько стадий выполнения первой инструкции. Это позволяет управляющим цепям процессора обрабатывать инструкции намного быстрее, чем при выполнении эксклюзивной полной обработки каждой инструкции от начала до конца.

\section{Описание алгоритмов}
В данной лабораторной работе предполагается создание программы на основе конвейрной обработки данных. В качестве алгоритмов на каждую из трех лент были выбраны следующие алгоритмы.
\begin{enumerate}
	\item Поиск среднего арифмитического значения в массиве.
	\item Подсчет количества числе в массиве больше, чем среднее значение.
	\item Определение является ли найденное количество простым числом или нет.
\end{enumerate}


\section{Вывод}
В данном разделе были рассмотрены основополагающие материалы, которые в дальнейшем потребуются для реализации алгоритмов. Были рассмотрены особенности построения конвейерных вычислений.

