\chapter{Конструкторская часть}
В данном разделе будут представлены схемы рассматриваемых алгоритмов и требования к вводу.

\section{Требования к вводу}
\begin{enumerate}
	\item На вход подаются два числа (первое - количество задач, второе - длина массива).
	\item Если на вход пришли не цифры, то программа завершается.
\end{enumerate}

\section{Схема алгоритма}

\subsection{Схема конве1ерной обработки данных}
Схема алгоритма главного потока представлена на рисунке \ref{img:s1}.

\img{150mm}{s1}{Схема алгоритма главного потока}

\clearpage

Схема алгоритма конвейера представлена на рисунке \ref{img:s2}.

\img{140mm}{s2}{Схема алгоритма конвейера}

\clearpage


\subsection{Схемы алгоритмов на лентах конвейера}

Схема нахождения среднего арифмитического значения массива представлена на рисунке \ref{img:s3}.


\img{150mm}{s3}{Схема нахождения среднего арифмитического значения массива}

\clearpage


Схема подсчета количества элементов в массиве, больших среднего арифметического массива, представлена на рисунке \ref{img:s4}.

\img{150mm}{s4}{Схема подсчета количества элементов в массиве, больших среднего арифметического массива}

\clearpage


Схема определения простоты числа представлена на рисунке \ref{img:s5}.

\img{150mm}{s5}{Схема определения простоты числа}


\section{Вывод}

В данном разделе были рассмотрены схемы однопоточной и
многопоточной реализации алгоритма ранговой сортировки.
