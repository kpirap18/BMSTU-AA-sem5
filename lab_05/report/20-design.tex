\chapter{Конструкторская часть}
В данном разделе будут представлены схемы рассматриваемых алгоритмов и требования к вводу.

\section{Требования к вводу}
Ниже описанные требования необходимо соблюдать для верной работы программы.

\begin{enumerate}
	\item На вход подаются два числа (первое -- количество задач, второе -- длина массива).
	\item Если на вход пришли не цифры, то программа завершается.
\end{enumerate}

\section{Разработка конвейера}

В данной программе конвейер состоит из 3 лент, каждая из которых выполняет соответствующую задачу. Для каждой ленты в главном потке создается отдельный рабочий поток.

Рабочий поток выполняется, пока не завершит обработку всех заявок.

Также в программе предусмотрен список обработанных задач и три очереди заявок - для каждой ленты. Очередь первой ленты заполняется заранее, во вторую заявки попадают после обработки на первой ленте, в третью очередь - после второй ленты, и в список обработанных заявок - после третьей ленты.

Стоит отметить тот факт, что ко 2 и 3 очередям могут одновременно обратиться сразу два потока: предыдущий (номер очереди - 1) для записи и текущий (соответствующий номеру очереди) для получения заявки. Поэтому для корректной работы с данными очередями следует из блокировать доступ для других потоков, для этого используются мьютексы (по одному для каждой очереди).

\section{Разработка алгоритмов}

\subsection{Алгоритм конвейерной обработки данных}
Конвейерная обработка данных выполняется потоком-диспетчером, или главным потоком, и набором рабочих потоков (лент), каждый из которых выполняет один вид подзадач и играет роль обработчика соответствующей стадии решения задачи. Рабочие потоки работают параллельно. 

На рисунке \ref{img:s1} представлена схема главного потока для параллельной конвейерной обработки, который запускает и контролирует рабочие потоки.

\img{200mm}{s1}{Схема главного потока}
\clearpage

На рисунке \ref{img:o1} представлена схема первого рабочего потока конвейера. 

\img{200mm}{o1}{Схема потока для поиска среднего значени в массиве}
\clearpage

На рисунке \ref{img:o2} представлена схема второго рабочего потока конвейера. 

\img{200mm}{o2}{Схема потока для подсчета элементов массива больших, чем среднее значение массива}
\clearpage

На рисунке \ref{img:o3} представлена схема третьего рабочего потока конвейера. 

\img{200mm}{o3}{Схема потока для определение того факта, простое ли число}
\clearpage


\subsection{Алгоритмы на лентах конвейера}

Схема нахождения среднего арифмитического значения массива представлена на рисунке \ref{img:s3}.


\img{150mm}{s3}{Схема нахождения среднего арифмитического значения массива}

\clearpage


Схема подсчета количества элементов в массиве, больших среднего арифметического массива, представлена на рисунке \ref{img:s4}.

\img{150mm}{s4}{Схема подсчета количества элементов в массиве, больших среднего арифметического массива}

\clearpage


Схема определения простоты числа представлена на рисунке \ref{img:s5}.

\img{150mm}{s5}{Схема определения простоты числа}


\section{Вывод}

В данном разделе были рассмотрены схемы конвейера и каждой его ленты, а также схемы разрабатываемых алгоритмов.
